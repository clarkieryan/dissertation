\chapter{Background \& Objectives}

\section{Background}

	With the cost of living going down people are looking further afield for evening entertainment, by not being from around the area they are at an immediate dis-advantage. With the increase of popularity of smart phones many people will pick up their smart phones, and immediately search for an application to help them find events on the go. The current selection of applications are very limited to the style of events and the area they are based, which means a new application for each time they go away this is not ideal and will quickly fill up a smart phone. 

	At this time there is a wide selection of similar applications available to download from the Apple AppStore. Each of these applications differ slightly be it the way the information is presented or the key functionality they provide. Many of the applications are only suited towards a particular city or specific headlining artists, by doing this they are somewhat limiting the scope of their audience. Many of the solutions out there also currently only utilise data that's being input by employees, this means that only a representative of all the events happening are selected and presented by the application. 

	\subsection{Current Solutions}
		Here are some of the current applications that are available to be downloaded from the Apple AppStore. All of the applications are free downloads, along with free registration to use the features. All of the applications as standard provide a feed of events (in a relevant order), and the ability to filter them in various ways. 

		\subsubsection{Line Up}
			Line Up shows a wide variety of events in the Manchester area, it gives you the ability to add events to their `Line Up' which is essentially a list of events that they are planning to go to or going to. It also gives you the opportunity to share the event via popular social network sites, increasing there own reach and allowing the users friends to see what they are attending/interested in. To discover events you are able to browse events by type of place, People, and all events. The application also allows you to follow other users that use the application, allowing a user to see what other users are attending. When viewing an individual event you can see a title, description, dates of the event, and the location. 
			% Add in screen shot of the application

		\subsubsection{Spotnight}
			Spotnight shows selected events happening in the London area, it gives you the ability to purchase tickets for the events available through a 3rd party service. You can view the events that are happening either this week or later on, however you are able to apply filters for specific areas, style of music, and genres. The application allows you to like events, however this does not seem to have any particular effect to the ordering of the events or any other aesthetic/functional item of the application. When viewing an individual event, you are able to view the location, description, price, images, people attending, and venue contact details. You are also able to share particular events through social networking sites; Facebook and Twitter. 
			% Add in screen shot of  the application
		
		\subsubsection{SongKick}
			SongKick offers a selection of artist specific events, 





What was your background preparation for the project? What similar systems did you assess? What was your motivation and interest in this project? 

\section{Analysis}

	

Taking into account the problem and what you learned from the background work, what was your analysis of the problem? How did your analysis help to decompose the problem into the main tasks that you would undertake? Were there alternative approaches? Why did you choose one approach compared to the alternatives? 

There should be a clear statement of the objectives of the work, which you will evaluate at the end of the work. 

In most cases, the agreed objectives or requirements will be the result of a compromise between what would ideally have been produced and what was felt to be possible in the time available. A discussion of the process of arriving at the final list is usually appropriate.

\section{Process}
You need to describe briefly the life cycle model or research method that you used. You do not need to write about all of the different process models that you are aware of. Focus on the process model that you have used. It is possible that you needed to adapt an existing process model to suit your project; clearly identify what you used and how you adapted it for your needs.

